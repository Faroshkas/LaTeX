\documentclass[a4paper]{article}
\usepackage{graphicx} % Required for inserting images
\usepackage{pgfplots}
\usepackage{geometry}

\title{Lista 2 - Cinemática}
\author{Francisco Dall' Oglio Scorsato}
\date{Abril 2024}
\pgfplotsset{compat=1.18}

\begin{document}

\setlength{\parindent}{20pt}
\maketitle



\section{Que tamanho deve ter um corpo para ser considerado um ponto material? Justifique.}
\hspace{\parindent}Um ponto material é todo corpo que irá modificar de maneira negligenciável o resultado do fenômeno em observação. Exemplo: \par
Ao observar um ecossistema, uma única formiga não fará grandes diferenças ao analisar o comportamento de leões. Na física, algo como um asteroide no sistema solar não irá modificar a atração gravitacional entre a Terra e o Sol.
\par Logo, não há um único tamanho específico para um corpo ser considerado um ponto material. Tudo depende do sistema observado e do efeito que o corpo tem nele.  

\section{Um carro percorre uma pista circular com raio de \(80m\). Determine o deslocamento e o espaço percorrido.}

\[\textrm{Deslocamento} = x - x_0\]
\[\textrm{Circunferência} = 2\times\pi\times r\]
\[= 2\times3\times80=480m\]
\[\textrm{Meio  Perímetro} =\frac{480}{2} = 240m\]

\subsection{a) }
\hspace{\parindent}O deslocamento total será a posição inicial subtraída da final. Ao dar uma volta completa em um círculo, o carro chegará de volta a onde começou. Portanto, o deslocamento total será \(x - x_0 = 0m\), visto que \(x=x_0\).
\par Já o espaço percorrido será toda a trajetória do automóvel. Foi dada uma volta completa pelo círculo, então esse espaço será igual à circunferência do círculo, ergo, \(480m\).

\subsection{b)} 
\hspace{\parindent}Ao dar meia volta em um círculo, o carro irá estar no exato lado oposto àquele que começou. Logo, sua distância de seu ponto de partida será o diâmetro do círculo, que, nesse caso, é \(2\times80=160m\). Esse é o deslocamento.
\par Já o espaço percorrido ao dar meia volta em um círculo será de meio perímetro. Isso é, \(240m\).


\section{Se a velocidade da luz no vácuo é \(300.000\frac{km}{s}\), qual é o valor de 1 ano-luz em quilômetros?}
\hspace{\parindent}1 ano-luz é o espaço que a luz percorre em um ano. Portanto, ao expressar a velocidade da luz em quilômetros por ano, é possível extrair tal distância.

\[\frac{300.000km}{s}=\frac{x}{ano}\]
\[ano=365\times dia\]
\[dia = 24\times h\]
\[h=60\times min\]
\[min = 60\times s\]
\[ano=365\times 24\times 60\times 60\times s\]
\[ano = 31.536.000s\]

\[\frac{300.000km}{s}=\frac{x}{31.536.000s}\]
\[x = 300.000\cdot 31.536.000\cdot km\]
\[x=9,4608\cdot 10^{12} km ft\]

\section{Um ônibus percorre \(180km\) entre 10:30h e 13:00h. Qual é a velocidade média do automóvel?}

\hspace{\parindent}O ônibus andou por 2:30h. Sua velocidade média é sua distância total percorrida dividida pelo tempo.
\[\frac{180km}{2,5h}=\frac{72km}{h}\]
\par Para transformar \(km/h\) em \(m/s\):
\[\frac{x\cdot km}{h} = \frac{x\cdot 1000m}{3600s} = \frac{\frac{x}{3,6}m}{s}\]
\hspace{\parindent}Logo:
\[\frac{72km}{h}=\frac{\frac{72}{3,6}m}{s}=\frac{20m}{s}\]

\section{Os dois automóveis realizam movimentos retilíneos constantes. Qual é a velocidade de B?}

\hspace{\parindent}Sabe-se que o veículo \(A\) levará o mesmo tempo para percorrer \(100m\) que o veículo \(B\) levará para percorrer \(60m\), afinal, os dois automóveis irão se encontrar, no mesmo instante, no mesmo ponto. \(A\), que anda \(10m\) a cada segundo, demorará \(10s\) para completar os \(100m\). Portanto, o carro \(B\) tem velocidade de: \[\frac{60m}{10s} = \frac{6m}{s}\] 
\section{Um corpo se move obedecendo a função horária \(x = 60 - 10t\).}

\begin{tikzpicture}
\begin{axis}[
axis lines = middle,
xlabel = \(t (s)\),
ylabel = {\hspace{\parindent}\(x (m)\)}
]
\addplot [
domain=0:8,
samples=100,
color=blue,
]
{60-10*x};
\addlegendentry{\(x = 60-10t\)}

\end{axis}
\end{tikzpicture}

\subsection{a)}
\subsubsection{Posição inicial}
\hspace{\parindent}A posição inicial do corpo é aquela quando \(t=0\). Substituindo na função:
\[60-10\cdot 0 = 60m\]
\par Isso está de acordo com a forma padrão da função, onde o termo constante indica a posição inicial.
\subsubsection{Velocidade}
\par Sua velocidade, por ser constante, será a mesma em todos os pontos. A forma padrão da função horária de posição retilínea uniforme nos diz que o coeficiente de \(t\) é a velocidade. Logo, a velocidade é \(10m/s\) no sentido negativo. Também é possível extrair esse fato da definição de velocidade:
\[\frac{\Delta x}{\Delta t}=-\frac{60m}{6s}=-\frac{10m}{s}\]

\subsection{b)}
\hspace{\parindent}Sua posição no instante 3s é aquela quando \(t=3\). Perceba:
\[60-10\cdot3=30m\]

\subsection{c)}
\hspace{\parindent}O instante em que passa pela origem das posições é quando seu deslocamento \(x\) é \(0\). Para descobrir isso, basta resolver a equação:
\[60-10t=0\rightarrow10t=60\rightarrow t=6s\]

\subsection{d)}
\hspace{\parindent}A distância percorrida em um intervalo de tempo por um corpo em velocidade constante é o intervalo de tempo multiplicado pela velocidade. Pode-se extrair esse fato da definição de velocidade:
\[v=\frac{\Delta x}{\Delta t}\]
\[\Delta x = v\cdot \Delta t\]
\par Nesse caso, de \(1s\) a \(10s\), o tempo decorrido é de \(9s\) e a velocidade é de \(-10m/s\). 
\par Então, a distância percorrida será de \(-90m\), ou \(90m\) no sentido negativo.


\section{A figura representa a posição de um corpo em função do tempo.}
\begin{tikzpicture}
\begin{axis}[
axis lines = middle,
xlabel = \(t (s)\),
ylabel = {\(x (m)\)},
xtick={0,25,50},
ytick={-50,-25,0,25,50},
]
\addplot [
domain=0:40,
samples=100,
color=blue,
]
{-50+2*x};

\end{axis}
\end{tikzpicture}

\subsection{a)}
\hspace{\parindent}Quando \(t=0\), o valor da função é de \(-50m\). Portanto, se \(x=m+bt\), \(m=-50\).
\par Quando \(t=25\), o valor da função é de \(0m\). Logo, \(-50+b\cdot(25)=0\).
\par Dessa equação, temos que \(b=2\) e a função horária é \(x=-50+2t\).

\subsection{b)}
\hspace{\parindent}O corpo passa pela posição \(80m\) quando \(x=80\).
\[x=-50+2t=80\]
\[2t=130\]
\[t=65s\]

\section{Um automóvel faz uma viagem com velocidade variada. Calcule a velocidade escalar média do veículo.}
\hspace{\parindent}A velocidade média de um veículo será toda a distância percorrida dividida por todo o tempo passado.
\[V_m = \frac{\Delta x}{\Delta t}\]
\par A distância pode ser adquirida calculando a área abaixo da curva do gráfico. Isso é:
\[\int_0^4{f(t)}dt=2\cdot80 + 0,5\cdot0+1,5\cdot40=220km\]
\par O tempo é dado, e é igual a 4 horas.
\par A velocidade média então é:
\[\frac{220km}{4h}=\frac{55km}{h}\]
\section{Qual é o tempo, em milissegundos, que o aparelho calcula ao medir um carro com velocidade de \(60km/h\)?}
\hspace{\parindent}O momento em que o tempo começa a ser medido é quando o veículo passa pelo sensor 1. Chame esse momento de \(t_0\). O relógio para de rodar quando o automóvel atravessa o sensor 2. Chame esse momento de \(t\). É possível montar as equações:
\[\Delta t=t-t_0\]
\[\Delta x = 0,5m\]
\[v=\frac{\Delta x}{\Delta t}=\frac{0,5m}{t-t_0}=\frac{60km}{h}\]
\[=\frac{60.000m}{3600000ms}= \frac{6m}{360ms} = \frac{1m}{60ms}\]

\par Se a cada \(60ms\), o carro percorre \(1m\), a cada \(30ms\), o carro percorrerá metade disso, vulgo \(0,5m\). Logo, o automóvel levará \(30ms\) para ir do sensor 1 ao sensor 2, e esse será o tempo medido pelo aparelho. Isso também pode ser mostrado algebricamente resolvendo a equação \(\frac{0,5}{\Delta t}=\frac{1}{60}\), que daria o mesmo resultado de \(30ms\).
\end{document}
Mudar:
Adicionar algo como "Logo, não há tamanho que defina um corpo material" na n. 1
Na n. 4, há um km no lugar de m
Adicionar a interpretação da n. 9
Melhorar algumas escritas, evitando repetições (portanto).
Botar aquele espaço antes do parágrafo (tab).