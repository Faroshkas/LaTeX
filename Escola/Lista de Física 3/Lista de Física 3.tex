\documentclass{article}
\usepackage{graphicx} % Required for inserting images
\usepackage{pgfplots}
\usepackage{geometry}

\title{Lista 3 - Cinemática}
\author{Francisco Dall' Oglio Scorsato}
\date{Abril 2024}
\pgfplotsset{compat=1.18}

\begin{document}

\maketitle

\section{Ao avistar um pedestre, o motorista reduz a velocidade de \(72km/h\) para \(36km/h\) em 4 segundos. Qual foi a aceleração, supondo que ela seja constante?}

    \[a=\frac{\Delta v}{\Delta t}\]
    \[\Delta v=36-72=-36km/h\]
    \[\Delta t=4s\]
    \[a=\frac{-36km/h}{4s}=-\frac{9000m/3600s}{s}\]
    \[a=-\frac{5}{2}m/s^2\] 

\section{É dado um movimento cuja função horária é \(x=-40-2t+2t^2\).}

    \begin{tikzpicture}
        \begin{axis}[
            axis lines = middle,
            xlabel = \(t (s)\),
            ylabel = \(x (m)\)
            ]
            \addplot [
                domain=0:8,
                samples=100,
                color=blue,
                ]
                {-40 - 2*x + 2*x^2};
            \addlegendentry{\(x =-40-2t+2t^2\)}
    
      \end{axis}
    \end{tikzpicture}

    \subsection{a)}
        \hspace{\parindent}Ao comparar a função horária da questão com a forma geral de funções horárias, podemos fazer as seguintes observações:
    
        \[x=x_0+v_0t+\frac{a}{2}t^2\]
        \[x=-40-2t+2t^2\]
        \[x_0=-40m\]
        \[v_0=-2m/s\]
        \[a=4m/s^2\]
    
    \subsection{b)}
        \hspace{\parindent}A função de velocidade escalar é a derivada da função de posição.
        \[v = \frac{d}{dt}[-40-2t+2t^2]=-2+4t\]
    
    \subsection{c)}
        \hspace{\parindent}O instante em que o corpo muda de sentido é quando a velocidade deixa de ser positiva e fica negativa ou vice-versa. Para descobrir isso, basta encontrar o ponto em que \(v=0\) e ver se esse é um ponto onde o sinal muda.
        \[v=-2+4t\]
        \[0=-2+4t\]
        \[t=\frac{1}{2}\]
        \par É trivial ver que, quando \(t<\frac{1}{2}\), \(v<0\) e que, quando \(t>\frac{1}{2}\), \(v>0\). Portanto, o instante em pauta é um válido candidato. \(t = 0,5s\).

\section{Um ponto parte do repouso com aceleração constante e após 10 segundos, encontra-se a \(40m\) da posição inicial.}

    \subsection{a)}
    \par Sabemos que \(v_0 = 0\) e \(x_0 = 0\). Com isso, temos:
        \[x(t)=\frac{a}{2}t^2 + v_0t + x_0=\frac{a}{2}t^2\]
        \[x(10)=40=\frac{a}{2}\cdot 100\]
        \[a=\frac{4}{5}=0,8m/s^2\]
    
    \subsection{b)}
        \[v(x)=at + v_0\]
        \[v(10)=0,8\cdot 10 + 0\]
        \[v=8m/s\]

\section{Um motorista atento aciona o freio à velocidade de 14,0m/s. O desatento, sob mesmas circunstâncias, aciona o freio um segundo depois. Que distância ele percorre a mais?} 

    \hspace{\parindent}A distância a mais que o desatento anda é igual a soma da distância que ele anda durante o segundo que ele levou para freiar e a distância que ele percorre freiando, subtraído da distância que o motorista atento percorre após freiar.

    \par A velocidade pós-freio do motorista atento é dada pela expressão:
    \[v=v_0+at=14-5t\]
    \par Ele irá parar quando sua velocidade for igual à zero. Isso é:
    \[14-5t=0 \rightarrow t=\frac{14}{5}\]
    \par Sua distância percorrida, sendo \(x_0\) sua posição inicial, é dada pela expressão:
    \[x=x_0+v_0t+\frac{a}{2}t^2=x_0+14t-\frac{5}{2}t^2\]
    \par A distância percorrida até o veículo parar é aquela quando \(t=\frac{14}{5}\).
    \[x=x_0+14\cdot\frac{14}{5}-\frac{5}{2}\cdot(\frac{14}{5})^2\]
    \[x=x_0+\frac{98}{5}=\frac{98}{5}=19,6m \textrm{   (Para os propósitos da questão, \(x_0 = 0\).)}\]
    \par A distância que o desatento percorreu antes de freirar é dado pela função de posição, com velocidade inicial de \(14m/s\):
    \[x = x_0 + v_0t +\frac{a}{2}t^2 = 0 + 14t + \frac{1}{2}t^2\]
    \par Após um segundo, que é quando ele começa a freiar, terá andado:
    \[x = x_0 + 14\cdot 1 + \frac{1}{2} \cdot 1^2 = 14,5m\]
    \par Podemos fazer um procedimento parecido ao que fizemos com o motorista atento para determinar a distância que o desatento percorreu após freiar. Sua velocidade no momento de freio é \(15m/s\), já que a aceleração é \(1m/s^2\).
    \[v=v_0+at=15-5t\]
    \[15-5t=0 \rightarrow t=3\]
    \[x=x_0+v_0t+\frac{a}{2}t^2=x_0+15t-\frac{5}{2}t^2\]
    \[x=x_0+15\cdot3-\frac{5}{2}\cdot3^2\]
    \[x=x_0+\frac{45}{2}=\frac{45}{2}=22,5m\]
    \par A distância total andada por ele será então \(14,5m + 22,5m = 37m\). 
    \par Para descobrir quanto a mais o motorista desatento andou, basta subtrair de seu deslocamento, o deslocamento do motorista atento. Isso é:
    \[37m - 19,6m = 17,4m\]

\section{Qual é a velocidade inicial de um trem com desaceleração de \(0,06m/s^2\) que desacelera completamente depois de 30km?}
    \[v^2=v_0^2+2a\Delta x\]
    \[0=v_0^2+2\cdot (-0,06)\cdot 30000\]
    \[v_0=60\]
    \par Logo, a velocidade com que o trem começa a desacelerar é de \(60m/s\), ou \(216km/h\).

\section{Qual é o gráfico certo?}
    \hspace{\parindent}O gráfico de posição com uma aceleração constante diferente de zero será uma parábola, com concavidade dependendente do sinal da aceleração. O gráfico de posição de uma velocidade constante será uma linha reta com inclinação diferente de zero. Com isso em mente, é possível descobrir qual gráfico descreve os movimentos da locomotiva.
    \par O primeiro terço do gráfico deve ter a forma de uma parábola côncava para cima. O segundo terço deve ser uma linha reta. E o terceiro deve ser parte de uma parábola côncava para baixo.
    \par A única opção que obedece todos os requisitos é a opção C.

\section{Dois automóveis, A e B, andam em uma estrada retilínea.}
    \hspace{\parindent}A velocidade de A é constante por todo o gráfico, e é igual à inclinação da reta. Isso é, \(v_A=\frac{\Delta x}{\Delta t}=10m/s\).
    \par Já a velocidade de B é variada. Descobrir esse valor em \(t=5s\) é o mesmo que descobrir a derivada de B em \(t=5s\).
    \[x(t)=x_0+v_0t+\frac{a}{2}t^2\]
    \[x(t)=0+0+\frac{a}{2}t^2\]
    \[x(5)=50=\frac{a}{2}5^2\]
    \[a=4m/s^2\]
    \[x'(t)=at=4t\]
    \[x'(5)=20m/s\]
    \[v_B=20m/s\]
    \[\frac{v_A}{v_B}=\frac{10m/s}{20m/s}\]
    \[\frac{v_A}{v_B}=\frac{1}{2}\]

\section{Considere o gráfico e avalie as afirmações, justificando as respostas.}
    \subsection{Nos intervalos de tempo de 2,0s a 4,0s e de 6,0s a 8,0s o corpo permanece em repouso.}
        \hspace{\parindent} Isso é falso. O gráfico representa a velocidade em função do tempo. Isso quer dizer que, durante os intervalos citados, a velocidade era constante e era diferente de zero, o que quer dizer que o corpo não estava em repouso.
    \subsection{De 0 até 8,0s só há um trecho de movimento uniformemente acelerado.}
        \hspace{\parindent} Isso é verdade. Esse trecho é aquele de 0s a 2,0s. O corpo acelera de maneira uniforme.
    \subsection{De 0 até 8,0s só há um trecho de movimento uniformemente retardado.}
        \hspace{\parindent} Isso é verdade. Esse trecho é aquele de 4,0s a 6,0s.
    \subsection{O afastamento máximo da origem do referencial é maior do que 40m.}
        \hspace{\parindent} Isso é falso. Para descobrir isso, basta observar qual é o momento em que o corpo percorreu a maior quantidade de espaço. Já que o gráfico é um de velocidade, o deslocamento é dado pela área abaixo da curva. O maior valor andado é de \(\int_0^5v(t)dt=35m\), o que é menor que 40m.
    \subsection{O corpo passa somente uma vez pela posição 30m.}
        \hspace{\parindent} Isso é falso. \(\int_0^4v(t)dt=30m\) e \(\int_0^6v(t)dt=30m\).

\section{O gráfico indica como varia o espaço de um móvel em função do tempo. Qual é a aceleração?}
    \[x(t)=x_0+v_0t+\frac{a}{2}t^2\]
    \[x(t)=3+v_0t+\frac{a}{2}t^2\]
    \[x(2)=3+2v_0+2a=-1\]
    \[v_0+a=-2\]
    \[x(4)=3+4v_0+8a=3\]
    \[v_0+2a=0\]
    $$ \left\{
        \begin{array}{lr}
            v_0+a=-2\\
            v_0+2a = 0
        \end{array}
    \right. $$
    \[a=2m/s^2\]

\section{Quanto ao movimento de um corpo lançado verticalmente para cima e submetido somente à ação da gravidade, é correto afirmar que:}
    \subsection{A velocidade do corpo no ponto de altura máxima é zero instantaneamente.}
    \hspace{\parindent}Isso é verdade. Esse é o ponto instantâneo no qual o corpo não está subindo nem caindo.
    \subsection{A velocidade do corpo é constante para todo o percurso.}
    \hspace{\parindent}Isso é falso. A gravidade está mudando a velocidade do corpo a todo momento.
    \subsection{O tempo necessário para a subida é igual ao tempo de descida, sempre que o corpo é lançado de um ponto e retorna ao mesmo ponto.}
    \hspace{\parindent}Isso é verdade. As parábolas se comportam de tal forma a fazer isso ser verdadeiro.
    \subsection{A aceleração do corpo é maior na descida do que na subida.}
    \hspace{\parindent}Isso é falso. A gravidade é sempre constante em todos os momentos, seja na descida ou na subida.
    \subsection{Para um dado ponto na trajetória, a velocidade tem os mesmos valores, em módulo, na subida e na descida.}
    \hspace{\parindent}Isso é verdade. Uma demonstração disso seria mostrar que uma parábola é simétrica em relação ao vértice.

\section{Um menino lança uma bola verticalmente para cima do nível da rua. Outra pessoa apanha a bola a 10m de distância, quando ela está a caminho do chão. Qual era a velocidade da bola quando ela foi apanhada?}
    \[x_0=0\]
    \[v_0=15\]
    \[a=-10m/s\]
    \[x=15t-5t^2\]
    \[10=15t-5t^2\]
    \[t^2-3t+2=0\]
    \[(t-1)(t-2)=0\]
    \[t\in\{1,2\}\]
    \par Escolhemos t=2 pois a bola estava caindo quando foi apanhada, portanto, deve ser o tempo posterior.
    \[v=v_0+at\]
    \[v=15-10t\]
    \[v=15-10\cdot 2\]
    \[v=-5m/s\]
    
\end{document}



na linha 88, analisar a folha de exercícios e explicar pq v_0 = 15 e não 14. Se não me engano é pq a aceleração era de 1m/s^2 e ele freiou um segundo depois. Mas aí tem que explicar ali.