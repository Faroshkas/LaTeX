\documentclass{article}
\usepackage{graphicx} % Required for inserting images
\usepackage{pgfplots} % For making the axes
\usepackage{geometry} % For plotting functions
\usepackage{amsmath, amsthm, amssymb, amsfonts}
\usepackage{thmtools}
\usepackage{setspace}
\usepackage{float}
\usepackage{hyperref}
\usepackage[utf8]{inputenc}
\usepackage[english]{babel}
\usepackage{framed}
\usepackage[dvipsnames]{xcolor}
\usepackage{tcolorbox}

\colorlet{LightGray}{White!90!Periwinkle}
\colorlet{LightOrange}{Orange!15}
\colorlet{LightGreen}{Green!15}

\declaretheoremstyle[name=Theorem]{thmsty}
\declaretheorem[style=thmsty,numberwithin=section]{theorem}
\tcolorboxenvironment{theorem}{colback=LightGray}

\declaretheoremstyle[name=Proposition,]{prosty}
\declaretheorem[style=prosty,numberlike=theorem]{proposition}
\tcolorboxenvironment{proposition}{colback=LightOrange}

\declaretheoremstyle[name=Principle,]{prcpsty}
\declaretheorem[style=prcpsty,numberlike=theorem]{principle}
\tcolorboxenvironment{principle}{colback=LightGreen}

\title{Questão 10, Aula 2, Encontro 2, Ciclo 2}
\author{Francisco Dall' Oglio Scorsato}
\date{29 de maio de 2024}
\pgfplotsset{compat=1.18}

\begin{document}

\maketitle

\section{Resolução por sistema de equações}
Suponha que a expressão que representa a quantidade de palitos até determinado andar \(x\) é uma função quadrática. Isso é confirmado após se usar mais termos para os polinômios (os de mais alto grau cancelam).
Portanto, temos que \[y=c+bx+ax^2\] Sabemos também, por contagem, que os pontos \((1, 5), (2,14), (3, 27)\) estão no gráfico. Com isso, podemos montar o sistema de equações (o qual resolverei por eliminação de Gauss):
\[\begin{bmatrix}
    1 & 1 & 1\\
    1 & 2 & 4\\
    1 & 3 & 9
\end{bmatrix}
\begin{bmatrix}
    c\\
    b\\
    a
\end{bmatrix}
=
\begin{bmatrix}
    5\\
    14\\
    27
\end{bmatrix}\]
\[\begin{bmatrix}
    1 & 1 & 1\\
    0 & 1 & 3\\
    0 & 0 & 2
\end{bmatrix}
\begin{bmatrix}
    c\\
    b\\
    a
\end{bmatrix}
=
\begin{bmatrix}
    5\\
    9\\
    4
\end{bmatrix}\]    
\[\begin{bmatrix}
    c\\
    b\\
    a
\end{bmatrix}
=
\begin{bmatrix}
    0\\
    3\\
    2
\end{bmatrix}\]
Assim, temos a função:
\[y=2x^2+3x\]
Igualando isso a 2024, temos que \(31<x<32\). Podemos raciocinar que isso quer dizer que \(x\) está na trigésima segunda escada.

\section{Resolução por sequência}
\[S={5,14,27,44,\dots}\]
The first time, it increases by 9. The second, by 13. Then, by 17. The amount it grows by grows by 4.
\[a_1 = 5\]
\[a_n = 1+4n+a_{n-1}\]

\end{document}