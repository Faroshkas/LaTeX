\documentclass{article}
\usepackage{graphicx} % Required for inserting images
\usepackage{pgfplots} % For making the axes
\usepackage{geometry} % For plotting functions

\title{Hotel de Hilbert - Decomposição de uma fração como a soma de duas frações.}
\author{Francisco Dall' Oglio Scorsato}
\date{11 de Maio 2024}
\pgfplotsset{compat=1.18}

\begin{document}



\maketitle

\section{Questão}

Uma fração unitária é uma fração da forma \(\frac{1}{n}\) onde \(n\) é um inteiro positivo. Note que a fração unitária \(\frac{1}{11}\) pode ser escrita como a soma de duas frações unitárias das três maneiras:
    \[\frac{1}{11}=\frac{1}{12}+\frac{1}{132}=\frac{1}{22}+\frac{1}{22}=\frac{1}{132}+\frac{1}{12}.\]
Há outras maneiras de decompor \(\frac{1}{11}\) como a soma de duas frações unitárias? De quantas maneiras pode se escrever \(\frac{1}{60}\) como a soma de duas frações unitárias? De maneira geral, de quantas maneiras a fração unitária \(\frac{1}{n}\) pode ser escrita como a soma de duas frações unitárias? Em outras palavras, quantos pares ordenados \((a,b)\) de inteiros positivos \(a, b\) existem de tal forma que
    \[\frac{1}{n}=\frac{1}{a}+\frac{1}{b}?\]

\section{Solução}
Seja \(\frac{1}{n}=\frac{1}{a}+\frac{1}{b}\). Disso temos:
\[\frac{1}{n}-\frac{1}{a}=\frac{1}{b}\rightarrow \frac{a-n}{an}=\frac{1}{b}\].
Para que \(\frac{a-n}{an}\) seja uma fração unitária, é necessário que \(a-n = 1\) ou que \(an\) seja divisível por \(a-n\). O motivo pelo segundo caso ser verdade é que, para transformar \(\frac{a-n}{an}\) em uma fração unitária, temos que dividir o numerador por \(a-n\) e, por consequência, o denominador também. Logo, o denominador deve ser divisível.
Vamos analisar o primeiro caso:

\subsection{\(a-n = 1\)}
Se \(a-n = 1\), segue que \(a=n+1\). Também segue que, já que a fração já é unitária, então seu denominador é \(b\). Portanto:
\[\frac{1}{n}=\frac{1}{a}+\frac{1}{b}=\frac{1}{n+1}+\frac{1}{(n+1)\cdot n}.\]
Esse é nosso primeiro par \((a,b)\): \(a = n+1\) e \(b = n(n+1)\). Se \((a,b)\) é um par, então \((b,a)\) também é. Portanto, descobrimos outro par, \((n(n+1),n+1)\). 

\subsection{\(an\) é divisível por \(a-n\)}
Sabemos dois fatores de \(an\): \(a\) e \(n\). Portanto, \(an\div (a-n)=a\) ou \(an\div (a-n)=n\).
No primeiro caso, temos que:
\[an\div (a-n)=a\rightarrow n\div (a-n) = 1\rightarrow n = (a-n)\rightarrow a=2n\]
Se \(a=2n\), então \(\frac{1}{n}=\frac{1}{2n}+\frac{1}{b}\rightarrow b=2n\)
Logo, outro par é \((2n,2n)\).

No segundo caso, temos que:
\[an\div (a-n)=n\rightarrow a\div (a-n) = 1\rightarrow a = (a-n)\rightarrow n=0\]
Já que \(\frac{1}{n}\) com \(n=0\) é indefinido, então esse caso não tem solução.

\section{Conclusão}
Os três possíveis pares \((a,b)\) são:
\((n+1),n(n+1)\), \((n(n+1),(n+1))\) e \((2n,2n)\). Pode se aplicar isso para \(n=11\) e \(n=60\) para responder as outras perguntas da questão.
Como um pequeno bônus, há uma quantidade infinita de formas de representar uma fração unitária como a soma de \(k\) frações unitárias. Basta somar \(\frac{1}{kn}\) consigo mesmo \(k\) vezes.

\end{document}